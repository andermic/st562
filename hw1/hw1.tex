\documentclass{article}
\usepackage[pdftex]{graphicx}
\usepackage{amsmath}
\usepackage{verbatim}
\usepackage{enumerate}
\author{Michael Anderson}
\title{Homework 1}
\begin{document}
\setlength{\parskip}{1em}
\maketitle
\center{ST562}
\flushleft
\newpage

\section{6.2.1}
\[
f_X(x) = p^2(1-p)^{x_1}(1-p)^{x_2} = p^2(1-p)^{x_1+x_2}
\]

\[
f_T(t) = \sum_0^t p(1-p)^i p(1-p)^{t-i} = p^2(t+1)(1-p)^t
\]

\[
\frac{f_X(x)}{f_T(t)} = \frac{p^2(1-p)^{x_1+x_2}}{p^2(t+1)(1-p)^{x_1+x_2}} =
\frac{1}{t+1}
\]

$\frac{f_X(x)}{f_T(t)}$ is not a function of $p$.

\section{6.2.2}
Substituting in $1-p$ for $q$, and letting $\hat x = \sum_0^m x_i$ and 
$\hat y = \sum_0^n y_i$ gives:
\[
f_{X,Y}(x,y) = \left( \prod_1^m p^{x_i}(1-p)^{1-x_i} \right)
\left( \prod_1^n p^{1-y_i}(1-p)^{y_i} \right) =
p^{n+\hat x - \hat y}(1-p)^{m-\hat x + \hat y}
\]

$\hat X$ is distributed $Bin(p,m), \hat Y$ is distributed 
$Bin(1-p,n), T = \hat X - \hat Y$. Using the discrete version of convolution for
subtracting random variables we have:

\[
f_T(t) = \sum_0^n f_{\hat Y}(\hat{y_i})f_{\hat X}(t+\hat{y_i}) =
\]

\[
\sum_0^n \left[ C^n_{\hat{y_i}} p^{n-\hat{y_i}}(1-p)^{\hat{y_i}}
C^m_{t+\hat{y_i}} p^{t+\hat{y_i}}(1-p)^{m-(t+y_i)}\right] =
\]

\[
p^{n+t}(1-p)^{m-t} \sum_0^nC^n_{\hat{y_i}}C^m_{t+\hat{y_i}}
\]

\[
\frac{f_{X,Y}(x,y)}{f_T(t)} = \frac{p^{n+\hat x - \hat y}(1-p)^{m-\hat x + 
\hat y}}{p^{n+t}(1-p)^{m-t} \sum_0^nC^n_{\hat{y_i}}C^m_{t+\hat{y_i}}} =
\frac{1}{\sum_0^nC^n_{\hat{y_i}}C^m_{t+\hat{y_i}}}
\]

$\frac{f_{X,Y}(x,y)}{f_T(t)}$ is not a function of $p$.

\section{6.2.5}
Using the resulting form of $f_{X,Y}(x,y)$ from the previous problem:
\[
f_{X,Y}(x,y) = p^{3+(x_1+x_2)-(y_1+y_2+y_3)}(1-p)^{2-(x_1+x_2)+(y_1+y_2+y_3)}
\]

$Y_1Y_2 = 1$ iff $Y_1 = Y_2 = 1$, so:

\[
f_{Y_1Y_2}(y) = (q^2)^y(1-q^2)^{1-y} = (1-p)^{2y} \left[ 1-(1-p)^2 \right]^{1-y}
\]

Now by discrete convolution, where $T = X_1+Y_1Y_2$ :

\[
f_T(t) = f_X(t)f_{Y_1Y_2}(0) + f_X(t-1)f_{Y_1Y_2}(1) = p^t(1-p)^{1-t}p^2 +
p^{t-1}(1-p)^{1-(t-1)}(1-p^2) =
\]

\[
p^{t+2}(1-p)^{1-t}+p^{t-1}(1-p)^{2-t}(1-p^2)
\]

\[
\frac{f_{X,Y}(x,y)}{f_T(t)} = \frac{p^{3+(x_1+x_2)-(y_1+y_2+y_3)}(1-p)^
{2-(x_1+x_2)+(y_1+y_2+y_3)}} {p^{t+2}(1-p)^{1-t}+p^{t-1}(1-p)^{2-t}(1-p^2)}
\]

Cannot cancel out $p$'s out of the above, $\frac{f_{X,Y}(x,y)}{f_T(t)}$ is a
function of $p$.

\section{6.2.6}
\[
f_{X,Y}(x,y) = \prod_1^m \frac{\lambda^{x_i}e^{-\lambda}}{x_i!}
\prod_1^n \frac{2^{y_i}\lambda^{y_i}e^{-2\lambda}}{y_i!} =
\frac{2^{(\sum_0^n y_i)}\lambda^{(\sum_0^m x_i) + (\sum_0^n y_i)}e^{-\lambda(m+2n)}}
{(\prod_1^m x_i!)(\prod_1^n y_i!)}
\]

Since $Poisson(\lambda_1) + Poisson(\lambda_2) = Poisson(\lambda_1+\lambda_2)$:

\[
f_T(t) = \frac{(\lambda[m+2n])^t \hspace{2pt} e^{-\lambda(m+2n)}}{t!}
\]

\[
\frac{f_{X,Y}(x,y)}{f_T(t)} = \frac{\frac{2^{(\sum_0^n y_i)}\lambda^{(\sum_0^m x_i) + (\sum_0^n y_i)}e^{-\lambda(m+2n)}}{(\prod_1^m x_i!)(\prod_1^n y_i!)}}
{\frac{(\lambda[m+2n])^t \hspace{2pt} e^{-\lambda(m+2n)}}{t!}} =
\frac{2^{(\sum_0^n y_1)}t!}{(\prod_1^m x_i!)(\prod_1^n y_i!)(m+2n)^t}
\]

$\frac{f_{X,Y}(x,y)}{f_T(t)}$ is not a function of $\lambda$.

\section{6.2.7}
\[
f_{X,Y}(x,y) = \frac{2^{(\sum_1^5 y_i)} \lambda^{(\sum_1^4 x_i) + (\sum_1^5 y_1)e^{-14\lambda}}}
{(\prod_1^4) x_i!)(\prod_1^5 y_i!)}
\]

\[
f_T(t) = \frac{(3\lambda)^t e^{-3\lambda}}{t!}
\]

\[
\frac{f_{X,Y}(x,y)}{f_T(t)} = \frac{\frac{2^{(\sum_1^5 y_i)} \lambda^{(\sum_1^4 x_i) + (\sum_1^5 y_1)e^{-14\lambda}}} {(\prod_1^4) x_i!)(\prod_1^5 y_i!)}}
{\frac{(3\lambda)^t e^{-3\lambda}}{t!}} =
\]

\[
\frac{2^{(\sum_1^5)}}{3^t} \lambda^{(\sum_2^m x_i) + (\sum_2^n y_i)}
e^{-11\lambda} \frac{t!}{(\prod_1^m x_i)(\prod_1^n y_i)}
\]

Cannot cancel $p$'s from above, $\frac{f_{X,Y}(x,y)}{f_T(t)}$ is a function of
$p$.

\section{6.2.10}
\begin{enumerate}[(i)]
\item
\[
f_X(x) = \frac{1}{(2\pi\sigma^2)^{n/2}} \hspace{2pt}
\text{exp} \left\{ -\frac{\sum_1^n(x_i-\mu)^2}{2\sigma^2} \right\} =
\]

\[
\frac{1}{(2\pi\sigma^2)^{n/2}} \hspace{2pt}
\text{exp} \left\{ -\frac{\sum_1^n(x_i^2-2x_i\mu+\mu^2)}{2\sigma^2} \right\} =
\]
 
\[
\frac{1}{(2\pi\sigma^2)^{n/2}} \hspace{2pt}
\text{exp} \left\{ -\frac{\sum_1^n x_i^2}{2\sigma^2} \right\}
\text{exp} \left\{ n\frac{2\mu n^{-1}\sum_1^n x_i -\mu^2}{2\sigma^2} \right\}
\]

\hspace{3em}

\item
\[
f_X(x) = \frac{1}{(2\pi\sigma^2)^{n/2}} \hspace{2pt}
\text{exp} \left\{ -\frac{nn^{-1}\sum_1^n(x_i-\mu)^2}{2\sigma^2} \right\}1
\]

\end{enumerate}

\section{6.2.11}
\begin{enumerate}[(i)]
\item
\[
f_X(x) = \sigma^{-n} \text{exp} \left\{ -\sum_1^n (x_i-\mu)/\sigma \right\}
I(x_{(1)}>\mu) = 
\]

\[
\sigma^{-n} \text{exp} \left\{ -\sum_1^n (x_i-\mu)/\sigma \right\}
\text{exp} \{ \mu n/\sigma \} I(x_{(1)}>\mu)
\]

\hspace{3em}

\item
\[
f_X(x) = \sigma^{-n} \text{exp} \left\{ -nn^{-1}\sum_1^n (x_i-\mu)/\sigma \right\}
I(x_{(1)}>\mu)
\]

\hspace{3em}

\item
\[
f_X(x) = \sigma^{-n} \exp \left\{ -\sum_1^n (x_i-\mu)/\sigma \right\}
I(x_{(1)}>\mu) =
\]

\[
\sigma^{-n} \exp \left\{ -\sum_1^n [(x_i-x_{(1)}) - (\mu - x_{(1)})]/\sigma \right\} I(x_{(1)}>\mu) =
\]

\[
\sigma^{-n} \exp \left\{ - \left[ \left(\sum_1^n x_i - x_{(1)} \right) - n\mu + nx_{(1)} \right] /\sigma \right\} I(x_{(1)}>\mu)
\]

\end{enumerate}

\section{6.2.13}
\[
f_X(x) = I \left( \theta - \frac{1}{2} < X_{(1)} \right)
I\left(X_{(n)} < \theta + \frac{1}{2} \right)1
\]

\section{6.2.16}
\[
f_X(x) = 2^n\theta^{-n} \exp \left\{ - \left( \sum_1^n x_i^2 \right) /\theta \right\} \left( \prod_1^n x_i \right) I(x_{(1)} > 0)
\]

\end{document}
