\documentclass{article}
\usepackage[pdftex]{graphicx}
\usepackage{amsmath}
\usepackage{verbatim}
\usepackage{enumerate}
\author{Michael Anderson}
\title{Homework 2}
\begin{document}
\setlength{\parskip}{1em}
\maketitle
\center{ST562}
\center{Prof. Di}
\flushleft
\newpage

\section{6.3.2}

\[
f_X(x) = \sigma^{-n} \exp \left\{-\sum_1^n (x_i-\mu)/\sigma \right\}
I(x_{(1)}>\mu)
\]

\begin{enumerate}[(i)]

\item
\[
f_X(x)/f_X(y) = \exp \left\{ \left( \sum_{i=1}^n y_i - \sum_{j=1}^n x_j \right)
/\sigma \right\}
\frac{I(x_{(1)}>\mu)}{I(y_{(1)}>\mu)}
\]

Let $A = I(x_{(1)}>\mu)/I(y_{(1)}>\mu)$. Suppose $x_{(1)} > y_{(1)}$.
Then consider the following three possible cases for the value of $\mu$
relative to the order statistics:

\[
\left\{ \begin{array}{l l}
A = 0/0 & \text{if } \mu > x_{(1)} \\
A = 1/0 & \text{if } x_{(1)} > \mu > y_{(1)} \\
A = 1 & \text{if } y_{(1)} > \mu \\
\end{array} \right.
\]

Since the above shows that in this instance the value of $A$,
and by extension the value of the entire likelihood
ratio, is dependent on $\mu$, $x_{(1)} \le y_{(1)}$ to make the likelihood
ratio not dependent on $\mu$. Now supposing $x_{(1)} < y_{(1)}$, a symmetric
argument to the above will show
that $x_{(1)} \ge y_{(1)}$. So $x_{(1)} = y_{(1)}$ to make the likelihood
ratio not depend on $\mu$, and so $x_{(1)}$ is minimally sufficient for
$\mu$. 

\item

Again:

\[
f_X(x)/f_X(y) = \exp \left\{ \left( \sum_{i=1}^n y_i - \sum_{j=1}^n x_j \right)
/\sigma \right\}
\frac{I(x_{(1)}>\mu)}{I(y_{(1)}>\mu)}
\]

$\sum_{j=1}^n x_i = \sum_{i=1}^n y_j$ in order to make $\sigma$ drop out.
Since

\[
n^{-1} \sum_1^n (x_i-\mu) = n^{-1} 
\left( \left[\sum_1^n x_i\right] - \mu n \right)
\]

and since $n$ and $\mu$ are known, $n^{-1} \sum_1^n(x_i - \mu$)
is a function of $\sum_1^n x_i$ and is minimally sufficient for $\sigma$.

\item

Again:

\[
f_X(x)/f_X(y) = \exp \left\{ \left( \sum_{i=1}^n y_i - \sum_{j=1}^n x_j \right)
/\sigma \right\}
\frac{I(x_{(1)}>\mu)}{I(y_{(1)}>\mu)}
\]

Clearly need $\sum_{i=1}^n x_i = \sum_{j=1}^n x_j$ and $x_{(1)} = y_{(1)}$ to
eliminate both $\mu$ and $\sigma$, making $(x_{(1)}, \sum_1^n x_i)$
minimally sufficient. Since

\[
\left(x_{(1)}, \sum_1^n[x_i-x_{(1)}] \right) = 
\left(x_{(1)}, \left[\sum_1^n x_i \right] - nx_{(1)} \right)
\]

$\left(x_{(1)}, \sum_1^n[x_i-x_{(1)}] \right)$ is a function of
$(x_{(1)}, \sum_1^n x_i)$ and is minimally sufficient.

\end{enumerate}

\section{6.3.8}

\[
f_X(x) = \frac{1}{(2\pi\theta^2)^{-n/2}}
\exp \left\{ - \sum_1^n(x_i-\theta)^2/(2\theta^2) \right\}
\]

\[
= \frac{1}{(2\pi\theta^2)^{-n/2}}
\exp \left\{ - \sum_1^n(x_i^2-2x_i\theta+\theta^2)/(2\theta^2) \right\}
\]

\[
= \frac{1}{(2\pi\theta^2)^{-n/2}}
\exp \left\{-\frac{\sum_1^n x_i^2}{2\theta^2} + \frac{2 \sum_1^n x_i\theta}
{2\theta^2} - \frac{n\theta^2}{2\theta^2} \right\}
\]


\[
= A(\theta) \exp \left\{-\frac{\sum_1^n x_i^2}{2\theta^2} +
\frac{\sum_1^n x_i}{\theta} \right\}
\]

A distribution with the given pdf is not a member of the 2-parameter
exponential family, because the parameter vector contains only one element.
It is however part of the \emph{curved exponential family}, as defined in the
class notes.

To apply Theorem 3 from the notes, let $R_1 = \sum_1^n x_i^2$, and
$\eta_1 = -1/(2\theta)$, and $R_2 = \sum_1^n x_i$, and $\eta_2 = \theta$. Since
$R_1$ and $R_2$ are linearly independent, we get the following minimal
sufficient statistic $T$:

\[
T = (R_1,R_2) = \left(\sum_1^n x_i^2, \sum_1^n x_i\right)
\]

\section{6.3.9}
\[
f_X(x) = \frac{1}{(2\pi\theta)^{-n/2}}
\exp \left\{ - \sum_1^n(x_i-\theta)^2/(2\theta) \right\}
\]

\[
= \frac{1}{(2\pi\theta)^{-n/2}}
\exp \left\{ - \sum_1^n(x_i^2-2x_i\theta+\theta^2)/(2\theta) \right\}
\]

\[
= \frac{1}{(2\pi\theta)^{-n/2}}
\exp \left\{-\frac{\sum_1^n x_i^2}{2\theta} + \frac{2 \sum_1^n x_i\theta}
{2\theta} - \frac{n\theta^2}{2\theta} \right\}
\]


\[
= A(\theta)h(x) \exp \left\{-\frac{\sum_1^n x_i^2}{2\theta^2} \right\}
\]

As in the last problem, this distribution is not a 2-parameter exponential.

Here Theorem 3 applies because the family is full rank. Let
$R_1 = \sum_1^n x_i^2$ and $\eta_1 = -1/(2\theta^2)$. Then the statistic
$T = \sum_1^n x_i^2$ is minimally sufficient.

\section{6.3.12}
\[
f_X(x) = (2\theta)^{-n} \ I(-\theta < x_{(1)} < \theta) \ 
I(-\theta < x_{(n)} < \theta)
\]

\[
f_X(x)/f_X(y) = \frac{I(-\theta < x_{(1)} < \theta) \ 
I(-\theta < x_{(n)} < \theta)}
{I(-\theta < y_{(1)} < \theta) \ I(-\theta < y_{(n)} < \theta)}
\]

\[
= \frac{I(|x_{(1)}| < \theta) \ I(|x_{(n)}| < \theta)}
{I(|y_{(1)}| < \theta) \ I(|y_{(n)}| < \theta)}
\]

\[
= \frac{I(\max(|x_{(1)}|,|x_{(n)}|) < \theta)}
{I(\max(|y_{(1)}|,|y_{(n)}|) < \theta)}
\]

$\theta$ disappears from the pdf when 
$\max(|x_{(1)}|,|x_{(n)}|) = \max(|y_{(1)}|,|y_{(n)}|)$, and it is clear that
there is no courser partition of the sample space, so 
$\max(|x_{(1)}|,|x_{(n)}|)$ is minimally sufficient for $\theta$.

\section{6.3.13}

\[
f_{X,Y}(x,y) = \frac{1}{(2\pi\sigma^2)^{(m+n)/2}}
\exp \left\{- \left[\sum_{i=1}^m(x_i-\mu_1)^2+\sum_{j=1}^n(y_j-\mu_2)^2\right] /(2\sigma^2)\right\}
\]

\[
=\frac{1}{(2\pi\sigma^2)^{(m+n)/2}}
exp \left\{ -\frac{\sum_{i=1}^m x_i^2 + \sum_{j=1}^n y_j^2}{2\sigma^2} +
\frac{\sum_{i=1}^m x_i\mu_1}{\sigma^2} + \frac{\sum_{j=1}^n y_j\mu_2}{\sigma^2}
 - \frac{\mu_1^2}{2\sigma^2} - \frac{\mu_2^2}{2\sigma^2} \right\}
\]

\[
= A(\theta) 
\exp \left\{ -\frac{\sum_{i=1}^m x_i^2 + \sum_{j=1}^n y_j^2}{2\sigma^2} +
\frac{\sum_{i=1}^m x_i\mu_1}{\sigma^2} + \frac{\sum_{j=1}^n y_j\mu_2}{\sigma^2}
\right\}
\]

This family is full rank, so apply Theorem 3. 
Let $R_1 = \sum_{i=1}^m x_i^2 + \sum_{j=1}^n y_j^2$, and
$R_2 = \sum_{i=1}^m x_i$, and $R_3 = \sum_{j=1}^n y_j$ to get the following
minimal sufficient statistic by Theorem 3:

\[
T = \left(\sum_{i=1}^m x_i^2 + \sum_{j=1}^n y_j^2, \ \ 
\sum_{i=1}^m x_i, \ \ \sum_{j=1}^n y_j\right)
\]

\section{6.3.15}

\[
f_{X,Y}(x,y) = 
[\beta^{\alpha}\Gamma(\alpha)]^{-m}[k\beta^{\alpha}\Gamma(\alpha)]^{-n}
\exp \left\{-\frac{\sum_{i=1}^n x_i}{\beta} 
-\frac{\sum_{j=1}^n y_j}{k\beta}\right\}
\prod_{i=1}^mx_i^{\alpha-1}\prod_{j=1}^ny_j^{\alpha-1}
\]

\[
= A(\theta) 
\exp \left\{ \left[-\frac{k\sum_{i=1}^n x_i + \sum_{j=1}^n y_j}{k\beta}
\right] + \log \left[ \prod_{i=1}^mx_i^{\alpha-1}\prod_{j=1}^ny_j^{\alpha-1}
\right] \right\}
\]

\[
= A(\theta) 
\exp \left\{ \left[-\frac{k\sum_{i=1}^n x_i + \sum_{j=1}^n y_j}{k\beta}
\right] + (\alpha-1)\log \left[ \prod_{i=1}^mx_i\prod_{j=1}^ny_j
\right] \right\}
\]

This family is full rank, so apply Theorem 3.
Factoring functions of $\theta$ out of each term in the exponent, and treating
$log$ as an invertible function can give the following minimal sufficient
statistic:

\[
T = \left( k\sum_{i=1}^n x_i + \sum_{j=1}^n y_j, 
\ \ \prod_{i=1}^mx_i\prod_{j=1}^ny_j \right)
\]

\section{6.3.18}

\[
f_{X,Y}(x,y) = (2\pi\sigma^2)^{-m/2}(2\pi k\sigma^2)^{-n/2}
\exp \left\{-\frac{\sum_{i=1}^m(x_i-\mu)^2}{2\sigma^2}
-\frac{\sum_{j=1}^n y_j^2}{2k\sigma^2} \right\}
\]

\[
= (2\pi\sigma^2)^{-m/2}(2\pi k\sigma^2)^{-n/2}
\exp \left\{-\frac{\sum_{i=1}^mx_i^2}{2\sigma^2} +
\frac{\mu\sum_{i=1}^m x_i}{\sigma^2} - \frac{m\mu^2}{2\sigma^2}
-\frac{\sum_{j=1}^n y_j^2}{2k\sigma^2} \right\}
\]

\[
= A(\theta) \exp \left\{-\frac{k\sum_{i=1}^mx_i^2+\sum_{j=1}^ny_i^2}{2k\sigma^2}
+ \frac{\mu\sum_{i=1}^m x_i}{\sigma^2} \right\}
\]

This family is full rank, so apply Theorem 3. Factoring functions of $\theta$
out of the terms
in the exponential gives the following minimally sufficient statistic:

\[
T = \left(k\sum_{i=1}^mx_i^2+\sum_{j=1}^ny_i^2, \ \ \sum_{i=1}^m x_i \right)
\]

\end{document}
